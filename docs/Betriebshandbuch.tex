\documentclass[]{article}
\usepackage{rotating}
\usepackage{lmodern}
\usepackage{amssymb,amsmath}
\usepackage{ifxetex,ifluatex}
\usepackage{fixltx2e} % provides \textsubscript
\ifnum 0\ifxetex 1\fi\ifluatex 1\fi=0 % if pdftex
  \usepackage[T1]{fontenc}
  \usepackage[utf8]{inputenc}
\else % if luatex or xelatex
  \ifxetex
    \usepackage{mathspec}
  \else
    \usepackage{fontspec}
  \fi
  \defaultfontfeatures{Ligatures=TeX,Scale=MatchLowercase}
\fi
% use upquote if available, for straight quotes in verbatim environments
\IfFileExists{upquote.sty}{\usepackage{upquote}}{}
% use microtype if available
\IfFileExists{microtype.sty}{%
\usepackage{microtype}
\UseMicrotypeSet[protrusion]{basicmath} % disable protrusion for tt fonts
}{}
\usepackage{hyperref}
\hypersetup{unicode=true,
            pdfborder={0 0 0},
            breaklinks=true}
\urlstyle{same}  % don't use monospace font for urls
\usepackage{longtable,booktabs}
\usepackage{graphicx,grffile}
\makeatletter
\def\maxwidth{\ifdim\Gin@nat@width>\linewidth\linewidth\else\Gin@nat@width\fi}
\def\maxheight{\ifdim\Gin@nat@height>\textheight\textheight\else\Gin@nat@height\fi}
\makeatother
% Scale images if necessary, so that they will not overflow the page
% margins by default, and it is still possible to overwrite the defaults
% using explicit options in \includegraphics[width, height, ...]{}
\setkeys{Gin}{width=\maxwidth,height=\maxheight,keepaspectratio}
\IfFileExists{parskip.sty}{%
\usepackage{parskip}
}{% else
\setlength{\parindent}{0pt}
\setlength{\parskip}{6pt plus 2pt minus 1pt}
}
\setlength{\emergencystretch}{3em}  % prevent overfull lines
\providecommand{\tightlist}{%
  \setlength{\itemsep}{0pt}\setlength{\parskip}{0pt}}
\setcounter{secnumdepth}{0}
% Redefines (sub)paragraphs to behave more like sections
\ifx\paragraph\undefined\else
\let\oldparagraph\paragraph
\renewcommand{\paragraph}[1]{\oldparagraph{#1}\mbox{}}
\fi
\ifx\subparagraph\undefined\else
\let\oldsubparagraph\subparagraph
\renewcommand{\subparagraph}[1]{\oldsubparagraph{#1}\mbox{}}
\fi

\date{}

\begin{document}

\section{Beschreibung der Systeme}\label{beschreibung-der-systeme}

\subsection{Dateiserver Nord}\label{dateiserver-nord}

\begin{longtable}{|l|p{6cm}|}
\toprule
Beschreibung & Stellt Dienste zum Austausch / Ablegen von Dateien im
Netzwerk bereit:

\begin{itemize}
\item
  zentrale Ablage von Dateien (Software, Dokumente, etc.)
\item
  zentrale Ablage von Backup-Dateien der anderen Services
\item
  SAMBA/SMB
\end{itemize}\tabularnewline
\midrule
Betriebsverantwortung & Gruppe 5\tabularnewline
Server- / Gerätebasis & 1 virtueller Server (Angemietet bei netcup
GmbH)\tabularnewline
Betriebssystem & Ubuntu 16.04\tabularnewline
Besonderheiten &\tabularnewline
Redundanzkonzept & Manuelle Redundanz über Dateiserver der Gruppe
10\tabularnewline
Geforderte Verfügbarkeit & Gemäß\tabularnewline
Liefert Funktionen für & \begin{itemize}
\item
  Zertifikatsserver (Backup)
\item
  Webserver (Backup)
\item
  Mailserver (Backup)
\item
  Netz (Backup)
\end{itemize}\tabularnewline
Benötigt Funktionen von & \begin{itemize}
\item
  Zertifikatsservern
\item
  Mailservern
\item
  Netz
\end{itemize}\tabularnewline
Systemverwaltung / Administration & Erfolgt via Ansible. Direkter Zugriff ausschließlich per SSH (Passwort oder Zertifikat)\tabularnewline
Backup &\tabularnewline
Restore &\tabularnewline
\bottomrule
\end{longtable}

\newpage

\subsection{Dateiserver Süd}\label{dateiserver-suxfcd}

\begin{longtable}{|l|p{6cm}|}
\toprule
Beschreibung & Stellt Dienste zum Austausch / Ablegen von Dateien im
Netzwerk bereit:

\begin{itemize}
\item
  zentrale Ablage von Dateien (Software, Dokumente, etc.)
\item
  zentrale Ablage von Backup-Dateien der anderen Services
\item
  Dienste: SAMBA
\end{itemize}\tabularnewline
\midrule
Betriebsverantwortung & Gruppe 10\tabularnewline
Server- / Gerätebasis & VServer CX20 von Hetzner\tabularnewline
Betriebssystem & Debian GNU/Linux 8.5\tabularnewline
Besonderheiten &\tabularnewline
Redundanzkonzept & Manuelle Redundanz über Dateiserver der Gruppe
5\tabularnewline
Geforderte Verfügbarkeit & Gemäß\tabularnewline
Liefert Funktionen für & \begin{itemize}
\item
  Zertifikatsserver (Backup)
\item
  Webserver (Backup)
\item
  Mailserver (Backup)
\item
  Netz (Backup)
\end{itemize}\tabularnewline
Benötigt Funktionen von & \begin{itemize}
\item
  Zertifikatsservern
\item
  Mailservern
\item
  Netz
\end{itemize}\tabularnewline
Systemverwaltung / Administration & Erfolgt via Ansible. Direkter Zugriff ausschließlich per SSH (Passwort oder Zertifikat)\tabularnewline
Backup &\tabularnewline
Restore &\tabularnewline
\bottomrule
\end{longtable}

\newpage

\section{Service Schnittstellen}\label{service-schnittstellen}

\section{Beschreibung der
Serviceprozesse}\label{beschreibung-der-serviceprozesse}

\subsection{Change Management}\label{change-management}

Das Change Management hat das Ziel, dass alle Anpassungen der
IT-Infrastruktur kontrolliert, effizient und unter Minimierung von
Risiken für den Betrieb bestehender Services durchgeführt werden.

Wir orientieren uns hier am Prozess aus der \textbf{IT Infrastructure
Library (ITIL)}. Aus Ressourcengründen wurde auch hier auf eine
pragmatische Lösung gesetzt - es wurden nur die Prozesse übernommen,
welche für einen sicheren Betrieb der Umgebung aus unserer Sicht
zwingend erforderlich sind. Auf eine Zertifizierung nach ISO/IEC
20000:2005 wurde daher verzichtet.

\subsubsection{Workflow}\label{workflow}

\begin{enumerate}
\def\labelenumi{\arabic{enumi}.}
\item
  \textbf{Request for Change (RfC):} Ein Mitarbeiter der vereinten
  Backwerke stellt beim Change Manager einen formalen RfC. Dafür wird
  das Formular 5.1.1 genutzt. Der RfC enthält alle Informationen, die
  für die weitere Entscheidungsfindung wichtig sind.
\item
  \textbf{Change Advisory Board (CAB):} Das CAB besteht aus dem Change
  Manager und von ihm benannten weiteren Personen. Es muss nach
  Beantragung innerhalb von 7 Tagen zusammentreten und den RfC
  genehmigen - oder diesen alternativ, inklusive Begründung der Ablehnung, an
  den Antragsteller zurück übermitteln.
\item
  \textbf{Implementierung}: Genehmigte Changes werden zur Bearbeitung an
  die entsprechenden technischen Gruppen weitergegeben.
\item
  \textbf{Post Implementation Review:} Ziel ist es, die Effizienz der
  durchgeführten Maßnahmen sowie des dazugehörigen Prozesses zu
  durchleuchten. Dabei sollen sowohl die durchgeführte Veränderung als
  auch die dabei benutzten Methoden und Prozesse einer Ist-Soll-Analyse
  unterzogen werden. Bei größeren Changes spielen auch
  Kosten-Nutzen-Vergleiche, die Return on Investment (ROI)-Kalkulation
  sowie die Messung der Zielerreichung aus der geschäftlichen
  Perspektive eine Rolle. Der Change Manager legt hierzu Termine und
  Agenda fest und fordert dafür Unterstützung an.
\end{enumerate}

\subsubsection{Change Manager}\label{change-manager}

Der Change-Manager ist verantwortlich für die Durchführung eines Changes
in einer systematischen Art und Weise, nachdem die bekannten Risiken
abgewogen wurden. Er überwacht auch den Fortschritt des Changes. Der
Change-Manager beurteilt die Requests for Change (RfC) zusammen mit dem
Change Advisory Board (CAB). Change Manager ist \textbf{Sascha
Girrulat}.

\subsubsection{Standard Changes}\label{standard-changes}

Bei Standard Changes handelt es sich um häufig wiederkehrende Changes
mit vergleichsweise geringem Risiko, deren Prozess vom CAB einmalig
genehmigt wurde.

Das Verfahren zur erstmaligen Genehmigung eines Standard Changes ist das
Gleiche wie bei herkömmlichen Changes. Wiederkehrende Standard Changes
werden nicht durch einen RfC geleitet, sondern lediglich als Service
Requests dokumentiert.

\paragraph{Genehmigte Standard Changes}\label{genehmigte-standard-changes}

\begin{longtable}{|p{3cm}|p{10cm}|}
\toprule
Standard Change & Prozess\tabularnewline
\midrule
Betriebssystem Update Server & \begin{enumerate}
\item
  Aufnahme als Service Request im Ticketsystem
\item
  Überprüfung des Changelogs auf Risiken
\item
  Einspielen des Updates außerhalb der garantierten Betriebszeit (siehe Seite
  11) oder Information aller Gruppen der Vereinigten Backwerke 48 Stunden vor
  Einspielen des Updates
\item
  Überprüfung des, durch den Server bereitgestellten, Services auf
  Funktion
\item
  Dokumentation im Ticketsystem.
\end{enumerate}\tabularnewline
\midrule

Dateiserver Dienst Update &
\begin{enumerate}
\item
  Aufnahme als Service Request im Ticketsystem
\item
  Überprüfung des Changelogs auf Risiken

\item
  Einspielen des Updates außerhalb der garantierten Betriebszeit (siehe Seite 11) oder Information aller Gruppen der Vereinigten Backwerke 48 Stunden vor Einspielen des Updates

\item
  Überprüfung des, durch den Server bereitgestellten, Services auf
  Funktion
\item
  Dokumentation im Ticketsystem.
\end{enumerate}\tabularnewline
Dateiserver:\\ Benutzerkonten lokal (einrichten, ändern, Passwort
zurücksetzen) & \begin{enumerate}
\item
  Änderung vornehmen in Ansible-Konfiguration
\item
  Einspielen der neuen Konfiguration auf beiden Servern
\item
  Betroffenen Nutzer über Änderungen informieren
\item
  Warten auf Rückmeldung des Nutzers (bei Einrichtung oder Änderung)
\end{enumerate}\tabularnewline
Offline Backup Dateiserver (1x wöchentlich, So) &
\begin{enumerate}
\item
  Prüfung, ob aktueller Termineintrag noch in gemeinsamem Kalender
  vorhanden
\item
  Login auf dem Mumble Server des CSIRT
\item
  Bekanntgabe an alle Anwesenden, dass mit der Sicherung begonnen wird
\item
  Erstellen eines Client-lokalen Backups von einem der beiden
  Dateiserver, inklusive des tar.gz-Backups vom anderen Server auf eine
  externe Festplatte

  \begin{enumerate}
  \def\labelenumii{\alph{enumii}.}
  \item
    Sicherung der eigentlichen Anwender-Dateien
  \item
    Sicherung der Systemeinstellungen mittels tdbbackup /etc/samba/passdb.tdb

    \begin{enumerate}
    \def\labelenumiii{\roman{enumiii}.}
    \item
      smb.conf
    \item
      secrets.tdb
    \item
      tdbsam
    \end{enumerate}
  \item
    Integrität der Backup-DB prüfen mittels tdbbackup -v etc/samba/passdb.tdb
  \item
    Sicherung der verwendeten Software: Samba
  \item
    Sicherung der Protokolldateien: rsyslog
  \end{enumerate}
\item
  Trennen der externen Festplatte vom Computer
\item
  Löschen des aktuellen Termineintrags im gemeinsamen Kalender
\item
  Logout auf dem Mumble Server
\end{enumerate}\tabularnewline
\midrule
Restore Dateiserver & \begin{enumerate}
\def\labelenumi{\arabic{enumi}.}
\item
  Ein Mitglied der Gruppe mit Betriebsverantwortung für den
  entsprechenden Sever verifiziert, dass dieser tatsächlich nicht mehr
  wieder in Betrieb genommen werden kann
\item
  Neuinstallation des Betriebssystems auf dem Gleichen oder, falls dies
  nicht mehr möglich ist, auf einem anderen Server
\item
  Einrichtung des SSH Zugangs auf dem neuen System
\item
  Restore der Serverkonfiguration auf letzte funktionierende
  Ansible-Version (ggf. Anpassung der externen IP-Adresse)
\item
  Einspielen der Nutzerdaten aus dem letzten verfügbaren Backup (vom
  anderen Dateiserver)
\item
  Mindestens zwei Techniker der Dateiservergruppen bestätigen die
  Wiederverfügbarkeit der Daten
\item
  Information an alle Anwender der Vereinigten Backwerke über
  vorgenommenen Restore sowie an Change und Problem Manager
\item
  Dokumentation im
  \href{https://docs.google.com/spreadsheets/d/1GDq3AEsVDu1a-X7tEl6qKDtSZdAXm8lsRzY6Ijw-dUQ/edit\#gid=0}{Ticketsystem}
\end{enumerate}\tabularnewline
\midrule
Benutzer verlässt das Unternehmen & \begin{enumerate}
\def\labelenumi{\arabic{enumi}.}
\item
  Aufnahme als Service Request im
  \href{https://docs.google.com/spreadsheets/d/1GDq3AEsVDu1a-X7tEl6qKDtSZdAXm8lsRzY6Ijw-dUQ/edit\#gid=0}{Ticketsystem}
\item
  Löschen ggf. vorhandener lokaler Konten auf den Dateiservern
\item
  Löschen der persönlichen Benutzerfreigaben auf beiden Dateiservern
\item
  Dokumentation im
  \href{https://docs.google.com/spreadsheets/d/1GDq3AEsVDu1a-X7tEl6qKDtSZdAXm8lsRzY6Ijw-dUQ/edit\#gid=0}{Ticketsystem}
\end{enumerate}\tabularnewline
\bottomrule
\end{longtable}

\subsubsection{Emergency Changes}\label{emergency-changes}

Ein Sonderfall des Changes, der nicht den üblichen Prozess durchläuft
sondern sofort, notfalls auch unter erheblichem Risiko und ohne weitere
Genehmigung vom CAB, meist zur Abwendung größeren Schadens durchgeführt
wird. \emph{Bei einem Change dieser Art benötigt der durchführende
Techniker nur die Zustimmung eines weiteren Technikers innerhalb der
IT-Gruppen.} Vorrangig ist hier auf eine Notsituation reagieren zu
können. Entsprechende weitere Genehmigungen und Tests werden erst nach
dem Change durchgeführt.

Die durchgeführten Emergency Changes sind jedoch umgehend nach der
Wiederherstellung des normalen Betriebes zu dokumentieren und die
betroffenen Gruppen sowie der Change Manager sind zeitnah zu
informieren.

\subsection{Incident Management}\label{incident-management}

Unter einem Incident / einer Störung versteht man nach IT Infrastructure
Library (ITIL) ein Ereignis, das nicht zum standardmäßigen Betrieb eines
Services gehört und das tatsächlich oder potenziell eine Unterbrechung
dieses Services oder eine Minderung der vereinbarten Qualität
verursacht.

IT-Incident Management umfasst den gesamten organisatorischen und
technischen Prozess der Reaktion auf solche Ereignisse. Ziel des
Incident-Management-Prozesses ist die schnellstmögliche
Wiederherstellung der Service-Leistung (auch mit Workarounds).

\subsubsection{Benutzerservice}\label{benutzerservice}

Um den Anwendern einen Single Point of Contact für alle IT-Anliegen zur
Verfügung zu stellen, haben wir einen Benutzerservice eingerichtet. Der
Benutzerservice hat die Aufgabe Anwenderanfragen entgegen zu nehmen, im
Ticketsystem anzulegen und das Ticket dann an die betroffenen Gruppen
weiter zu vermitteln.

\subsubsection{Workflow}\label{workflow-1}

\begin{enumerate}
\def\labelenumi{\arabic{enumi}.}
\item
  \textbf{Meldung} des Incidents an:\\
  (Mitarbeiter des Benutzerservice können hier direkt mit 2. weiter
  machen)
\item
  \textbf{Aufnahme} und Bewertung des Incident im
  \href{https://docs.google.com/spreadsheets/d/1GDq3AEsVDu1a-X7tEl6qKDtSZdAXm8lsRzY6Ijw-dUQ/edit\#gid=0}{Ticketsystem}
\item
  Falls die Anfrage direkt beantwortet/gelöst werden kann weiter mit 6.
  - sonst:\\
  Information per Mail an alle Mitglieder der betroffenen Gruppen
\item
  \textbf{Bearbeitung} des Incident in den Gruppen
\item
  Dokumentation des Bearbeitungsstandes im
  \href{https://docs.google.com/spreadsheets/d/1GDq3AEsVDu1a-X7tEl6qKDtSZdAXm8lsRzY6Ijw-dUQ/edit\#gid=0}{Ticketsystem}
\item
  Nach Abschluss der Arbeiten

  \begin{enumerate}
  \def\labelenumii{\alph{enumii}.}
  \item
    Information an den Anfrager sowie aller direkt betroffenen Anwender
  \item
    \textbf{Dokumentation} der Lösung im
    \href{https://docs.google.com/spreadsheets/d/1GDq3AEsVDu1a-X7tEl6qKDtSZdAXm8lsRzY6Ijw-dUQ/edit\#gid=0}{Ticketsystem}
  \end{enumerate}
\end{enumerate}

\subsubsection{IT Security Incident
Management}\label{it-security-incident-management}

IT Sicherheitsmanagement beinhaltet die Überwachung und Feststellung von
Sicherheitsereignissen auf dem System und die angemessene Reaktion auf
diese Ereignisse.

\paragraph{Computer Security Incident Response Team
(CSIRT)}\label{computer-security-incident-response-team-csirt}

\includegraphics[width=4.83401in,height=2.85457in]{images/image3.png}

Die Aufgaben des CSIRT sind

\begin{itemize}
\item
  die Überwachung der Systeme auf Sicherheitsvorfälle
\item
  als zentraler Kommunikationsknoten für eingehende
  Sicherheitsereignisse und ausgehende Informationen über diese
  Ereignisse zu agieren
\item
  Sicherheitsvorfälle zu dokumentieren
\item
  Das Bewusstsein für sichere IT innerhalb der Vereinigten Backwerke zu
  schärfen, um Vorfällen vorzubeugen
\item
  Externe System- und Netzwerkaudits zu unterstützen (etwa bei der
  Schwachstellenanalyse und Penetrationstests)
\item
  sich über neue Angriffsvektoren und -strategien zu informieren
\item
  Informationen über Sicherheitsupdates eingesetzter Software einzuholen
  und diese an die Administratoren der Services zu verbreiten
\item
  Administratoren bei Sicherheitsfragen zu beraten
\item
  bei Sicherheitsvorfällen zeitnah und angemessen die technische
  Reaktion der Vereinigten Backwerke zu koordinieren und diese
  umzusetzen
\end{itemize}

Jedes Teammitglied ist eigenständig verantwortlich für

\begin{itemize}
\item
  die Bereithaltung wichtiger Software-Werkzeuge und eines vom
  Vereinigten Backwerke Netz unabhängigen Computers mit
  Internet-Anschluss
\item
  die Ablage wichtiger Notfallinformationen (IP-Adressen der Systeme,
  Passwörter, Betriebshandbuch, Kopien von CA Schlüsseln) offline an
  einer für denjenigen gut zugänglichen, sicheren Stelle
\item
  Die Überprüfung des
  \href{https://www.buerger-cert.de/archive?type=WIDTechnicalWarning}{Bürger-CERT}
  auf aktuelle Sicherheitswarnmeldungen
\end{itemize}

\subparagraph{Mitglieder}\label{mitglieder}
\begin{longtable}{ll}
\toprule
Aus Gruppe & Personen\tabularnewline
\midrule
Gruppe 5 (Datei Nord) & Christoph Weißenborn\\
                      & Jörg Ricardo Schumacher\tabularnewline
Gruppe 10 (Datei Süd) & Sascha Girrulat\\
                      & Silas Jansen\tabularnewline
\bottomrule
\end{longtable}

\subparagraph{Mumble Server}\label{mumble-server}

Bei einem Security Incident koordinieren sich die Mitglieder des CSIRT
über den Ersten der im folgenden aufgelisteten Server. Falls dieser
nicht verfügbar sein sollte, so koordinieren sie sich über den zweiten
Server, usw.

\begin{longtable}{lp{6cm}}
\toprule
Server & Login\tabularnewline
\midrule
Gruppe 10 & IP/DNS: 78.46.200.193

User: \textless{}beliebig\textgreater{}

Passwort: juXe1goi\tabularnewline
Gruppe 5 & IP: 5.45.103.136:64738

DNS: praktikum.ehanse.de:64738

User: \textless{}beliebig\textgreater{}

Passwort: piratenwilddiebe\tabularnewline
Last Resort & Skype, Google Hangout oder Telefonkonferenz

(Achtung, hier keine Geschäftsgeheimnisse kommunizieren!)\tabularnewline
\bottomrule
\end{longtable}

\paragraph{Workflow}\label{workflow-2}

\begin{enumerate}
\def\labelenumi{\arabic{enumi}.}
\item
  \textbf{Meldung} des Sicherheitsvorfalls durch Anwender, Administrator
  oder Monitoring an den Benutzerservice
\item
  Der Benutzerservice erstellt ein \textbf{Ticket} im
  \href{https://docs.google.com/spreadsheets/d/1GDq3AEsVDu1a-X7tEl6qKDtSZdAXm8lsRzY6Ijw-dUQ/edit\#gid=0}{Ticketsystem}
  und nimmt eine \textbf{Erstbewertung} des Vorfalls vor. Falls es sich
  nach dieser Bewertung um einen möglichen Security Incident (gemäß )
  handelt, werden alle Mitglieder des CSIRT (per Mail und - soweit
  angegeben - telefonisch) informiert. Sie erhalten dabei auch die ID
  des Tickets (Zeilennummer im Ticketsystem).
\item
  Das CSIRT koordiniert die Reaktion über die Mumble Server. Dabei
  orientiert sich das Team an den folgenden Leitlinien:

  \begin{enumerate}
  \def\labelenumii{\alph{enumii}.}
  \item
    Schaden begrenzen und weitere Risiken minimieren
  \item
    Die Art und den Umfang der Kompromittierung feststellen
  \item
    Beweise sichern
  \item
    Weitere Stellen informieren, falls notwendig (CIO, Polizei, BSI,
    etc.)
  \item
    Systeme wiederherstellen
  \item
    Informationen für die Dokumentation zusammenstellen und
    strukturieren
  \item
    Schaden und Kosten abschätzen
  \item
    Reaktion überprüfen und Verfahren anpassen
  \end{enumerate}
\end{enumerate}

\subsection{Problem Management}\label{problem-management}

Über das Problem-Management werden unbekannte Ursachen für tatsächliche
und potentielle Störungen (Incidents) innerhalb der IT-Services
untersucht und die Behebung gesteuert. Anders als das Incident
Management arbeitet das Problem-Management sowohl reaktiv als auch
proaktiv. Ein wesentliches Ziel ist hierbei die 'dauerhafte
Problemlösung'.

Ein Problem ist so lange offen, wie seine Ursache und deren Behebung
nicht erledigt sind; demgegenüber kann ein Incident geschlossen werden
sobald die vom Anwender benötigte Funktionalität wieder hergestellt ist.

Die Ursachen für das Problem werden analysiert und Maßnahmen zu ihrer
Verhinderung oder Behebung entwickelt. Ergebnis dieser Analyse ist
entweder ein Known Error (also die nun bekannte Ursache für eine
Störung) oder ein Workaround (Umgehungslösung).

Im Problem-Management erarbeitete Workarounds können dem Incident
Management zur Verfügung gestellt werden, um künftig für identische
Störungen eine schnelle Wiederherstellung des betroffenen IT-Services zu
ermöglichen.

Lösungen für bekannte Fehler (Known Errors) können als
Änderungsanforderung (Request for Change) an das Change Management
weitergeleitet werden.

Problems werden als solche im
\href{https://docs.google.com/spreadsheets/d/1GDq3AEsVDu1a-X7tEl6qKDtSZdAXm8lsRzY6Ijw-dUQ/edit\#gid=0}{Ticketsystem}
dokumentiert.

\subsubsection{Problem Manager}\label{problem-manager}

Der Problem Manager ist dafür verantwortlich, alle Problems über ihren
gesamten Lebenszyklus zu verwalten. Seine vorrangigen Ziele bestehen
darin, der Entstehung von Incidents vorzubeugen und die negativen
Auswirkungen von Incidents, die nicht verhindert werden können,
möglichst gering zu halten. Zu diesem Zweck pflegt er die Informationen
zu Known Errors und Workarounds.

\paragraph{Festgelegte Problem
Manager}\label{festgelegte-problem-manager}

\begin{longtable}{ll}
\toprule
Gruppe & Problem Manager\tabularnewline
\midrule
Gruppe 5 (Dateiserver) & Waldemar Schmidt\tabularnewline
Gruppe 10 (Dateiserver) & Silas Jansen\tabularnewline
\bottomrule
\end{longtable}

\subsection{Service Management}\label{service-management}

\subsubsection{\texorpdfstring{\protect\hypertarget{ux5fRef456699502}{}{\protect\hypertarget{ux5fToc457467702}{}{}}Service
Level Agreement
(SLA)}{Service Level Agreement (SLA)}}\label{service-level-agreement-sla}

Der Begriff Service-Level-Agreement (SLA) bezeichnet die Schnittstelle
zwischen Auftraggeber und Dienstleister für wiederkehrende
Dienstleistungen. Ziel ist es, die Kontrollmöglichkeiten für den
Auftraggeber transparent zu machen indem zugesicherte
Leistungseigenschaften, wie etwa Leistungsumfang, Reaktionszeit und
Schnelligkeit, der Bearbeitung genau beschrieben werden. Wichtiger
Bestandteil ist hierbei die Dienstgüte (Servicelevel), welche die
vereinbarte Leistungsqualität beschreibt.

\begin{longtable}{lp{6cm}}
\toprule
Serviceparameter & Geforderter Servicelevel\tabularnewline
\midrule
\protect\hypertarget{GarantierteBetriebszeit}{}{}Garantierte
Betriebszeit & Mo-Sa, 04:00-22:00 Uhr\tabularnewline
Verfügbarkeit & mind. 99,0\%

(bezogen auf den Kalendermonat)

Die max. Ausfalldauer pro Monat beträgt 7:10 Stunden.\tabularnewline
Reaktionszeit & Security Incident: 2 Tage

Incident (kritisch): 2 Tage

Incident (normal): 2 Arbeitstage

Problem: 14 Tage

Infoanfrage: 2 Arbeitstage

Service Request: 14 Tage\tabularnewline
Entstörungszeit & Security Incident: 2 Tage

Incident (kritisch): 3 Tage

Incident (normal): 4 Arbeitstage

Problem: -

Infoanfrage: -

Service Request: -\tabularnewline
\bottomrule
\end{longtable}

\paragraph{Klassifizierung von Events}\label{klassifizierung-von-events}

\begin{longtable}{lp{6cm}}
\toprule
Event & Beschreibung\tabularnewline
\midrule
Security Incident & Die Vertraulichkeit oder Integrität von Gütern,
Informationen, Daten und IT-Services der Vereinigten Backwerke ist
möglicherweise kompromittiert.\tabularnewline
Störung (kritisch) & Ausfall eines Services

\emph{oder}

erheblicher Teilausfall eines Services\tabularnewline
Störung (normal) & Geringer Teilausfall eines Services oder einer
Komponente

(z.B. Nichtverfügbarkeit von Zusatz-Funktionen)

\emph{oder}

Ausfall einer Redundanzkomponente

\emph{oder}

Nichtverfügbarkeit eines Services für einen einzelnen
Anwender\tabularnewline
Problem & Unerwünschtes oder unerwartetes Verhalten eines Services, das
aktuell ohne wesentliche Auswirkungen auf die Anwender
ist\tabularnewline
Infoanfrage & Informationsanfrage eines Anwenders, die ohne Änderung an
den Systemen geklärt werden kann\tabularnewline
Service Request & Anfrage eines Anwenders zur Bereitstellung eines neuen
Services

\emph{oder}

Anfrage eines Anwenders zur Bereitstellung einer bisher nicht
realisierten Funktionalität eines bestehenden Services\tabularnewline
\bottomrule
\end{longtable}

\subsection{Eskalationsprozedur}\label{eskalationsprozedur}

Eskalationen dienen in der Regel dazu, eine Störungsbearbeitung durch
Einbeziehung höherer Instanzen zu beschleunigen, bzw. durch eine
Erweiterung der Kompetenzen und Befugnisse zu ermöglichen.

Eskaliert wird sobald sich abzeichnet, dass die definierte Entstörzeit
nicht eingehalten werden kann - oder spätestens, wenn sie überschritten
wurde. An weitere Eskalationsebenen wird eskaliert, wenn die jeweilige
Eskalationsebene mit ihren Kompetenzen und Befugnissen nicht mehr
ausreichend zur Behebung der Störung beitragen kann.

Da Problems, Infoanfragen und Service Requests keiner definierten
Entstörzeit unterliegen, für ihre Lösung jedoch ggf. dennoch eine
Eskalation erforderlich ist, sollte diese - soweit möglich -
einvernehmlich vorgenommen werden.

\begin{longtable}{lp{6cm}}
\toprule
Eskalationsstufe & Verantwortlicher Ansprechpartner\tabularnewline
\midrule
\endhead
Normalfall & \textbf{Benutzerservice}

Tel.: +49 228-3875 8003

Mail: gruppe5@mayerbrot.intern\tabularnewline
1. Eskalation & \textbf{Problem Manager}

siehe Festgelegte Problem Manager\tabularnewline
2. Eskalation & \textbf{CIO}

Ralf Naues

Tel.: NA

Mail: k1599@fernuni-hagen.de\tabularnewline
\bottomrule
\end{longtable}

\subsection{Service Manager}\label{service-manager}

Für die Umsetzung des Servicemanagements wird ein IT Service Manager
eingesetzt. Für alle Belange der Anwender, die für die Steuerung der
Serviceleistungen relevant sind, steht der Service Manager zur
Verfügung. Er leistet in dieser Funktion jedoch keinen operativen
Service und ist keine Eskalationsinstanz bei akuten Vorfällen. Er
überwacht die Einhaltung der SLAs. Wichtige Informationsquellen hierfür
sind Einträge im Ticketsystem sowie persönliches Feedback von
Leitungsbereich, Anwendern und IT Gruppen.

Service Manager ist

\textbf{Christoph Weißenborn}

Tel.: +49151-12045346

Mail: christoph.weissenborn@live.de

\section{Beschreibung der
Betriebsprozesse}\label{beschreibung-der-betriebsprozesse}

\subsection{Einmalige Betriebsprozesse (ausgelöst durch
Event)}\label{einmalige-betriebsprozesse-ausgeluxf6st-durch-event}

Komplett bekannte, wiederkehrende Betriebsprozesse, welche Änderungen an
den Systemen erfordern, sind als Genehmigte Standard Changes
beschrieben. Bei den folgenden Prozessen handelt es sich daher nur um
allgemeine Leitlinien für im Vorfeld nicht bekannte Maßnahmen:

\subsubsection{Allgemeiner Change
Dateiserver}\label{allgemeiner-change-dateiserver}

\begin{itemize}
\item
  Nach jeder Änderung an der Datei smb.conf muss zunächst mit dem
  Programm testparm geprüft werden, ob die Syntax der
  Konfigurationsdatei korrekt ist. Syntaxfehler in der
  Konfigurationsdatei können sonst dazu führen, dass der Server nicht
  neu startet oder Sicherheitslücken entstehen.
\end{itemize}

\subsubsection{Einbindung externer Programme in SAMBA
(Dateiserver)}\label{einbindung-externer-programme-in-samba-dateiserver}

\begin{itemize}
\item
  Es ist sicherzustellen, dass nur Programme, die keine schadhafte
  Funktion besitzen von Samba aufgerufen werden. Mit dem Kommando\\
  user\textgreater{} testparm -vs \textbar{} grep -E "(command
  =)\textbar{}(script =)\textbar{}(exec =)\textbar{}\textbackslash{}
  (panic action =)\textbar{}(program =)"\\
  werden alle Parameter ausgegeben, die für die Einbindung externer
  Programme in Samba verantwortlich sind. Zusätzlich zu den Parametern
  werden die momentan gültigen Werte angezeigt.
\end{itemize}

\subsection{Zyklische Reviews}\label{zyklische-reviews}

\subsubsection{Service Review}\label{service-review}

\begin{longtable}{lp{6cm}}
\toprule
Zyklus & 1x im Quartal\tabularnewline
\midrule
Verantwortlich & Service Manager\tabularnewline
Teilnehmer & \begin{itemize}
\item
  Service Manager
\item
  Problem Manager
\item
  ein Mitglied jeder am Betriebshandbuch teilnehmenden IT Gruppe
\item
  ein Mitglied des CSIRT
\item
  weitere Administratoren auf Anfrage des Service Managers
\end{itemize}\tabularnewline
Agenda & \begin{itemize}
\item
  Einhaltung der SLAs
\item
  Anzahl und Status der Incidents im Betrachtungszeitraum
\item
  Entwicklung der Servicequalität
\item
  geplante (oder sich im Ablauf ergebende) Änderungen am Service
  Management
\end{itemize}\tabularnewline
\bottomrule
\end{longtable}
\newpage
\subsubsection{IT Security Review}\label{it-security-review}

\begin{longtable}{lp{6cm}}
\toprule
Zyklus & Jährlich\tabularnewline
\midrule
Verantwortlich & CIO\tabularnewline
Teilnehmer & \begin{itemize}
\item
  CIO
\item
  Alle Mitglieder des CSIRT
\item
  weitere Administratoren auf Anfrage des CIO
\item
  Geschäftsleitung auf Anfrage des CIO
\end{itemize}\tabularnewline
Agenda & \begin{itemize}
\item
  Wirksamkeit und Aktualität von Regelungen zur Datensicherheit
\item
  Wirksamkeit und Aktualität von Regelungen zum Datenschutz
\item
  Review von Sicherheitsvorfällen im Betrachtungszeitraum
\item
  Überprüfung der Einhaltung gesetzlicher Vorschriften (Lizensierung,
  etc.)
\item
  \href{https://www.buerger-cert.de/archive?type=WIDTechnicalWarning}{Bürger-CERT}
  Meldungen und Bearbeitung im letzten Jahr
\item
  Organisation einer IT Sicherheitskampagne alle 2-3 Jahre
\end{itemize}\tabularnewline
\bottomrule
\end{longtable}

\subsubsection{Datensicherung Review}\label{datensicherung-review}

\begin{longtable}{lp{6cm}}
\toprule
Zyklus & 1x im Quartal\tabularnewline
\midrule
\endhead
Verantwortlich & Alle Mitarbeiter Gruppe 5 und 10\tabularnewline
Teilnehmer & \begin{itemize}
\item
  Gruppe 5
\item
  Gruppe 10
\item
  Ein Mitarbeiter des CSIRT
\end{itemize}\tabularnewline
Agenda & \begin{itemize}
\item
  Test Restore aus dem Backup
\item
  Funktionierte das Backup im Betrachtungszeitraum wie definiert?
\item
  Volumina und Backupzeiten noch im Rahmen des Datensicherungskonzeptes?
\end{itemize}\tabularnewline
\bottomrule
\end{longtable}

\protect\hypertarget{ux5fToc457467708}{}{}

\section{Anlagen}\label{anlagen}

\subsection{Formulare}\label{formulare}

\subsubsection{Request for Change (RfC)}\label{request-for-change-rfc}

\begin{longtable}{|p{6cm}|p{6cm}|}
\toprule
Betroffener Service &\tabularnewline
\midrule
Grund für den Change: & \tabularnewline
Voraussichtliche Auswirkungen: &\tabularnewline
Rollbackszenario für den Fall eines Fehlschlags: &\tabularnewline
Priorität: &\tabularnewline
Beteiligte IT Gruppen: &\tabularnewline
Beteiligte Anwendergruppen: &\tabularnewline
Vorhersehbare Risiken: &\tabularnewline
Zeitplan zur Durchführung: &\tabularnewline
\bottomrule
\end{longtable}

\subsubsection{Aufnahme eines Services / Gruppe in
Betriebshandbuch}\label{aufnahme-eines-services-gruppe-in-betriebshandbuch}

\begin{longtable}{lp{6cm}}
\toprule
Bezeichnung des Services* &\tabularnewline
\midrule
Serverdaten* & Beschreibung:

Server- / Gerätebasis:

Betriebssystem:

Besonderheiten:

Redundanzkonzept:

Geforderte Verfügbarkeit:

Systemverwaltung / Administration:

Backup:

Restore:\tabularnewline
Service Management & Wollt ihr am gemeinsamen SLA teilnehmen?

(Unser Vorschlag: Verfügbarkeit Mo-Sa, 04:00-22:00 Uhr,
99\%)\tabularnewline
Incident Management & Der Benutzerservice nimmt Mails entgegen, trägt
sie ins bereitgestellte Ticketsystem ein und informiert die betroffenen
Gruppen über das neue Ticket. Wollt ihr am gemeinsamen Benutzerservice
teilnehmen?

Welche Gruppenmitglieder nehmen am Benutzerservice teil?\tabularnewline
IT Security Incident Management & Das Computer Security Incident
Response Team (CSIRT) reagiert gemeinsam auf Sicherheitsvorfälle und
behandelt diese angemessen. Wollt ihr am gemeinsamen CSIRT teilnehmen?

Welche Gruppenmitglieder nehmen am CSIRT teil?\tabularnewline
Problem Management & Wollt ihr am gemeinsamen Problem Management
teilnehmen?

Wer wird für eure Gruppe Problem Manager?\tabularnewline
Change Management & Wollt ihr am gemeinsamen Change Management (nach
ITIL) teilnehmen?

Möchtet ihr dafür jemanden als Change Manager benennen?

Welche Standard Changes möchtet ihr mit aufnehmen lassen? (Mit
Ablaufbeschreibung)\tabularnewline
\bottomrule
\end{longtable}

Mit * gekennzeichnete Informationen sind Pflicht. Alle anderen Angaben
optional.

\subsection{}\label{section}
\newpage
\subsection{Technische Informationen}\label{technische-informationen}

\subsubsection{Liste der IP Adressen und DNS-Namen pro
Gerät}\label{liste-der-ip-adressen-und-dns-namen-pro-geruxe4t}
\begin{sidewaystable}
\begin{longtable}{lllll}
\toprule
Service & Intern IP & Intern DNS & Extern IP & Extern DNS\tabularnewline
\midrule
Nord VPN/DNS & 10.8.1.1 /24 & VPNNord.MayerBrot.intern & &\tabularnewline
Nord Web & 10.8.1.20 /24 & WebNord.MayerBrot.intern & &\tabularnewline
Nord File & 10.8.1.30 /24 & FileNord.MayerBrot.intern & 5.45.103.136 &
praktikum.ehanse.de\tabularnewline
Nord CA & 10.8.1.40 /24 & CaNord.MayerBrot.intern & 46.101.99.119 &
ca-nord.fachpraktikum-1599.de\tabularnewline
Nord Mail & 10.8.1.50 /24 & MailNord.MayerBrot.intern & &\tabularnewline
Süd VPN & 10.8.3.1 /24 & & &\tabularnewline
Süd Web & /24 & & &\tabularnewline
Süd File & 10.8.3.14 /24 & fileserver.mueller-backwaren.de &
138.201.175.250&
static.250.175.201.138.clients.your-server.de\tabularnewline
Süd CA & 10.8.3.26 /24 & ca.mueller-backwaren.de & &
caserv-mueller.westeurope.cloudapp.azure.com\tabularnewline
Süd Mail & /24 & & &\tabularnewline
\bottomrule
\end{longtable}

\end{sidewaystable}
\subsection{Organisatorische
Informationen}\label{organisatorische-informationen}

\subsubsection{Technisch relevante Gesetze und
Verordnungen}\label{technisch-relevante-gesetze-und-verordnungen}

\begin{itemize}
\item
  \href{http://www.gesetze-im-internet.de/tkg_2004/index.html}{Telekommunikationsgesetz
  (TKG)}
\item
  \href{http://www.gesetze-im-internet.de/tk_v_2005/index.html}{Telekommunikationsüberwachungsverordnung
  (TKüV)}
\item
  \href{http://www.gesetze-im-internet.de/urhg/index.html}{Urheberrechtsgesetz
  (UrhG)}
\item
  \href{https://www.gesetze-im-internet.de/tk_v_2005/index.html\#BJNR313600005BJNE001301308}{Verordnung
  über die technische und organisatorische Umsetzung von Maßnahmen zur
  Überwachung der Telekommunikation}
\item
  \href{http://www.gesetze-im-internet.de/elektrog/index.html}{Elektro-
  und Elektronikgerätegesetz (ElektroG)}
\item
  \href{http://www.gesetze-im-internet.de/sigg_2001/}{Signaturgesetz
  (SigG)}
\end{itemize}

\end{document}
