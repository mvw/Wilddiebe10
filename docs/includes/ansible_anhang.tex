\section{Ansible}
\subsection{Testfälle}
\begin{itemize}
  \item Run syntax-checks via makefile
  \item Run the role/playbook via makefile
  \item Run the role/playbook again to make sure it's idempotent
  \item Check if rsyslog port 10514 is open
  \item Check if samba port 445 is open
  \item Check if ssh port 22 is open
  \item Check expected users will be created
  \item Check users will member of expected groups
  \item Check users will be member of the expected primary group
  \item Check if user homes got the expected permissions
  \item Check if groupuser homes got the expected permissions
  \item Check expected groups will be created
  \item Check if ssh service is up and running
  \item Check if samba service is up and running
  \item Check if samba group share is present
  \item Check if rsyslog service is up and running
  \item Check if nmbd service is disabled
  \item Check if openntpd service is up and running
  \item Check if quotas are enabled
  \item Check if quota limits are set as expected
  \item Check if the doc files are building.
\end{itemize}
\subsection{Ansible Rollen}
\subsubsection{Eigene Rollen}
\begin{itemize}
  \item k1599\_anti\_virus
  \item k1599\_common
  \item k1599\_file\_server
  \item k1599\_openvpn\_client
  \item k1599\_quota
  \item k1599\_rsyslog\_client
  \item k1599\_rsyslog\_server
  \item k1599\_ssh
  \item k1599\_time\_sync
  \item k1599\_users
\end{itemize}
\subsubsection{Externe Rollen}
\begin{itemize}
  \item ansible-role-firewall
\end{itemize}
\subsection{Erzeugte Gruppen}
\begin{itemize}
  \item web-nord
  \item mail-nord
  \item netz-nord
  \item cert-nord
  \item file-nord
  \item web-sued
  \item mail-sued
  \item netz-sued
  \item cert-sued
  \item file-sued
  \item fapra1599
  \item sshlogin
\end{itemize}
\subsection{Erzeugte Benutzer}
\begin{itemize}
  \item msiepmann
  \item shartmann
  \item tgrosswendt
  \item netz-nord
  \item fkamfenkel
  \item cpempelforth
  \item mklein
  \item file-sued
  \item sgirrulat
  \item fhofmann
  \item ahacker
  \item nnapp
  \item mwoerkom
  \item sbruch
  \item bfischer
  \item cboettge
  \item jherold
  \item fschweisfurth
  \item web-nord
  \item jwichert
  \item mkutter
  \item phaebel
  \item cert-nord
  \item ooffenburger
  \item mploeger
  \item dtroeger
  \item nreusch
  \item istieglitz
  \item cweissenborn
  \item lrichter
  \item jschumacher
  \item netz-sued
  \item swyes
  \item mail-sued
  \item msandkuehler
  \item emueller
  \item twinkelhorst
  \item mail-nord
  \item sganswind
  \item bwalter
  \item pholtkamp
  \item chesseling
  \item mschrenk
  \item wpankraz
  \item file-nord
  \item ksteinkohl
  \item tbayer
  \item mberner
  \item klauter
  \item web-sued
  \item wlindemann
  \item tschroeder
  \item msalwitzek
  \item jaffenzeller
  \item wschmidt
  \item aszewc
  \item mmueller
  \item sjansen
  \item cert-sued
\end{itemize}
\subsection{Firewall}
\subsubsection{Offene Ports - Gruppe Nord}
\begin{itemize}
  \item all: 22
  \item tun0: 445
  \item tun0: 10514
  \item all: 64738
\end{itemize}
\subsubsection{Offene Ports - Gruppe Sued}
\begin{itemize}
  \item all: 22
  \item tun0: 445
  \item tun0: 10514
\end{itemize}
\subsection{Einträge /etc/hosts}
\begin{itemize}
  \item 10.8.3.1 : vpn-s vpn.mueller-backwaren.de
  \item 10.8.3.14 : file-s fileserver.mueller-backwaren.de
  \item 10.8.3.18 : mail-s mail.mueller-backwaren.de mail.mueller-backwaren.tpweb.de
  \item 10.8.3.22 : web-s web.mueller-backwaren.de
  \item 10.8.3.26 : ca-s ca.mueller-backwaren.de
  \item 10.8.1.1 : vpn-n vpn.mayer-brot.de vpnnord.mayerbrot.intern
  \item 10.8.1.20 : web-n web.mayer-brot.de webnord.mayerbrot.intern
  \item 10.8.1.30 : file-n file.mayer-brot.de filenord.mayerbrot.intern
  \item 10.8.1.40 : ca-n ca.mayer-brot.de canord.mayerbrot.intern
  \item 10.8.1.50 : mail-n mail.mayer-brot.de mailnord.mayerbrot.intern
  \item 10.8.2.1 : gateway-n
  \item 10.8.2.2 : gateway-s
\end{itemize}

\section{Testprotokoll}
\captionof{table}{Identifizierte Maßnahmen}
\label{tab:massnahmen}
\begin{longtable}{p{6.8cm}p{2.4cm}p{2.4cm}p{3cm}}
\toprule
Funktion & Status Nord & Status Süd & Bemerkungen \\
\midrule
\textbf{Korrektheit und Vollständigkeit der Ansible Konfiguration }
\begin{itemize}
\item
  Server komplett zurücksetzen und Ansible Konfiguration durchführen
\item
  läuft durch ohne Fehler
\end{itemize} & OK & OK & \\
\midrule
\textbf{Einbinden Netzlaufwerk}

\begin{itemize}
\item
  Windows 8.1+ Client
\item
  AD Anmeldung
\item
  Transport via AES
\end{itemize} & & & \\
\midrule
\textbf{Daten schreiben auf Freigabe}

\begin{itemize}
\item
  Linux (Samba 4.x) Client
\end{itemize}

\begin{itemize}
\item
  Transport via AES
\end{itemize} & & & \\
\midrule
\textbf{Daten lesen von Freigabe}

\begin{itemize}
\item
  Windows 8.1+ Client
\item
  Transport via AES
\end{itemize} & & & \\
\midrule
\textbf{Einbinden Netzlaufwerk "Jeder"}

\begin{itemize}
\item
  Linux (Samba 4.x) Client
\end{itemize} & & & \\
\midrule
\textbf{Schreiben auf Jeder Freigabe }

\begin{itemize}
\item
  Windows 8.1+ Client
\item
  AD Anmeldung
\item
  Transport via AES
\end{itemize} & & & \\
\midrule
\textbf{Quotas}

\begin{itemize}
\item
  Windows 8.1+ Client
\item
  AD Anmeldung
\item
  Schreiben von \textgreater{}100 MB
\end{itemize} & & & \\
\midrule
\textbf{Quotas Jeder }

\begin{itemize}
\item
  Linux (Samba 4.x) Client
\item
  Schreiben von \textgreater{}100 MB
\end{itemize} & & & \\
\midrule
\textbf{Virus schreiben}

\begin{itemize}
\item
  EICAR-Testdatei auf Freigabe schreiben
\item 
  Eintrag von ClamAV im lokalen Log: Eicar-Test-Signature
\item 
  Eintrag im rsyslog auf anderem Server
\item
  Zugriff wird verhindert
\end{itemize} & & & \\
\midrule
\textbf{Zugriff via ipv6 nicht möglich}

\begin{itemize}
\item
  Zugriff von interner IP auf SMB Freigabe per IPv6
\item
  Zugriff von externer IP auf SMB Freigabe per IPv6
\end{itemize} & & & \\
\midrule
\textbf{Automatisches Backup}

\begin{itemize}
\item
  anlegen von Referenzdatei auf persönlicher Freigabe
\item
  anlegen von Referenzdatei auf Jeder-Freigabe
\item
  Überprüfung der Backup-Datei auf dem jeweils anderen Server nach 24h
\end{itemize} & & & \\
\midrule
\textbf{Syslog}

\begin{itemize}
\item
  Anmeldung mit Admin-Account auf Server
\item
  Ausführen von SUDO-ping
\item
  Anmeldung und ping sind im lokalen Log vorhanden
\item
  Anmeldung und ping sind im remote-log vorhanden
\end{itemize} & & & \\
\midrule
\textbf{Gesicherter Login}

\begin{itemize}
\item
  Test-Login mit Admin-Account und falschem Passwort wird nach max. 10
  Versuchen abgelehnt
\item
  Zeitpunkt des letzten Login und Logout werden beim Anmelden mitgeteilt
\end{itemize} & & & \\

\bottomrule
\end{longtable}