\documentclass[]{article}
\usepackage{lmodern}
\usepackage{amssymb,amsmath}
\usepackage{ifxetex,ifluatex}
\usepackage{fixltx2e} % provides \textsubscript
\ifnum 0\ifxetex 1\fi\ifluatex 1\fi=0 % if pdftex
  \usepackage[T1]{fontenc}
  \usepackage[utf8]{inputenc}
\else % if luatex or xelatex
  \ifxetex
    \usepackage{mathspec}
  \else
    \usepackage{fontspec}
  \fi
  \defaultfontfeatures{Ligatures=TeX,Scale=MatchLowercase}
\fi
% use upquote if available, for straight quotes in verbatim environments
\IfFileExists{upquote.sty}{\usepackage{upquote}}{}
% use microtype if available
\IfFileExists{microtype.sty}{%
\usepackage{microtype}
\UseMicrotypeSet[protrusion]{basicmath} % disable protrusion for tt fonts
}{}
\usepackage{hyperref}
\hypersetup{unicode=true,
            pdfborder={0 0 0},
            breaklinks=true}
\urlstyle{same}  % don't use monospace font for urls
\IfFileExists{parskip.sty}{%
\usepackage{parskip}
}{% else
\setlength{\parindent}{0pt}
\setlength{\parskip}{6pt plus 2pt minus 1pt}
}
\setlength{\emergencystretch}{3em}  % prevent overfull lines
\providecommand{\tightlist}{%
  \setlength{\itemsep}{0pt}\setlength{\parskip}{0pt}}
\setcounter{secnumdepth}{0}
% Redefines (sub)paragraphs to behave more like sections
\ifx\paragraph\undefined\else
\let\oldparagraph\paragraph
\renewcommand{\paragraph}[1]{\oldparagraph{#1}\mbox{}}
\fi
\ifx\subparagraph\undefined\else
\let\oldsubparagraph\subparagraph
\renewcommand{\subparagraph}[1]{\oldsubparagraph{#1}\mbox{}}
\fi

\date{}

\begin{document}

\section{Einführung}\label{einfuxfchrung}

Durch die zunehmende Nutzung von IT für Geschäftsprozesse in Unternehmen
, hängt der Geschäftserfolg zunehmend auch von der sicheren Nutzung
unserer IT-Systeme ab. Gemeinsam mit ihnen möchten wir schwere
Verluste für das Unternehmen abwenden, wie z.B.

\begin{itemize}
\item
  Durch einen Virenbefall können wichtige Geschäftsdaten
  unwiederbringlich verloren gehen.
\item
  Durch einen unsicheren Netzwerkzugang wird es Wirtschafspionen - die
  heutzutage von den zahlreichen Auslandsgeheimdiensten ihrer jeweiligen
  Länder unterstützt werden - ermöglicht, zentrale Geschäftsgeheimnisse
  zu entwenden und sich dadurch die von uns mühsam erarbeiteten und
  erforschten Ergebnisse zu eigen zu machen. Stellen sie sich vor,
  unsere unnachahmlichen Stutenwecken würden morgen aus China zum halben
  Preis geliefert werden!
\item
  In Folge eines unsachgemäßen Gebrauchs der Speicherumgebung haben
  Betrüger Zugriff auf unsere Kontodaten erhalten und so einen
  sechsstelligen Betrag ins Ausland überwiesen.
\end{itemize}

In unserer Netzwerkumgebung ist es jedem Mitarbeiter gestattet, mit
seinem eigenen Computer auf die IT-Services zuzugreifen (BYOD). Das
bedeutet auch, dass wir nur mit ihrer Hilfe dafür sorgen können, dass
wir auch morgen alle noch ein Unternehmen haben, das unsere Kunden -
und so auch unsere Familien - ernährt.

\section{Was muss ich tun?}\label{was-muss-ich-tun}

\begin{itemize}
\item
  \textbf{M 2.138 Strukturierte Datenhaltung}

  \begin{itemize}
  \item
    Speichern Sie Programm- und Arbeitsdaten in getrennten
    Verzeichnissen. Dies erleichtert die Übersichtlichkeit und
    Datensicherung.
  \item
    Richten Sie für verschiedene Aufgaben und Projekte getrennte
    Verzeichnisse ein.
  \item
    Legen Sie so wenig Dateien, wie möglich in personenbezogenen
    Verzeichnissen ab. Nutzen Sie statt dessen Gruppenablagen oder die
    Ablage für den allgemeinen Zugriff.
  \end{itemize}
\item
  \textbf{M 2.160 Regelungen zum Schutz vor Schadprogrammen}

  \begin{itemize}
  \item
    Aufgrund der dezentralen Verwaltung der Clients, ist jeder Anwender
    für ein aktuelles Virenschutz-Programm auf seinem Client selbst
    verantwortlich.
  \item
    Melden sie Schadprogramm-Infektionen dem Benutzerservice.
  \end{itemize}
\item
  \textbf{M 2.224 Vorbeugung gegen Schadprogramme}

  \begin{itemize}
  \item
    Alle von Dritten erhaltenen Dateien und Programme sollten vor der
    Aktivierung auf möglicherweise enthaltene Schadprogramme überprüft
    werden.
  \item
    Daten und Programme sollten grundsätzlich nur von vertrauenswürdigen
    Quellen geladen werden. Also von den
    Original-Web-Seiten oder Original-Datenträgern des Erstellers.
  \item
    Arbeiten Sie mit einem Benutzerkonto ohne lokale
    Administrator-Rechte.
  \item
    Stellen Sie alle Programme, welche Makros ausführen können (MS
    Office, Adobe Acrobat, etc.) so ein, dass diese Makros nicht
    automatisch ausgeführt werden können.
  \end{itemize}
\item
  \textbf{M 3.21 Sicherheitstechnische Einweisung der Telearbeiter}

  \begin{itemize}
  \item
    Halten Sie dienstliche und private Daten konsequent getrennt.
  \end{itemize}
\item
  \textbf{M 5.155 Datenschutz-Aspekte bei der Internet-Nutzung}

  \begin{itemize}
  \item
    Konfigurieren Sie ihren Browser so, dass er Cookies aus dem Internet
    nicht automatisch akzeptiert.
  \item
    Konfigurieren Sie ihren Browser so, dass er die Historie nach jeder
    Sitzung automatisch löscht. Führen Sie dienstliche und private
    Nutzung nicht über die gleiche Sitzung durch oder nutzen Sie für
    dienstliche und private Zwecke verschiedene Browser.
  \item
    Speichern Sie im Browser keine dienstlich sensiblen Angaben, wie
    Passwörter, Unternehmens-Kredikartendaten, Telefonnummern usw.
  \end{itemize}
\item
  \textbf{M 5.51 Sicherheitstechnische Anforderungen an die
  Kommunikationsverbindung Telearbeitsrechner - Institution}

  \begin{itemize}
  \item
    Achten Sie beim Surfen im Intranet auf eine abgesicherte Verbindung.
    (Schloss-Symbol im Browser)
  \end{itemize}
\item
  \textbf{M 2.46 Geeignetes Schlüsselmanagement}

  \begin{itemize}
  \item
    Verwenden Sie für jedes Login ein anderes Passwort.
  \item
    Verwenden Sie KeePass zum Managen Ihrer Passwörter und um neue
    sichere Passwörter zu erzeugen.
  \item
    Ändern Sie ihre verwendeten Passwörter mindestens einmal im Quartal.
  \item
    Besteht der Verdacht, dass eines Ihrer Passwörter kompromittiert
    wurde, informieren Sie bitte den Benutzerservice.
  \end{itemize}
\item
  \textbf{M 4.448 Einsatz von Verschlüsselung für Speicherlösungen}

  \begin{itemize}
  \item
    Verschlüsseln Sie Daten, die einen hohen Schutzbedarf bezüglich
    Vertraulichkeit aufweisen, mit einem aktuellen
    Verschlüsselungsprogramm (etwa VeraCrypt) auf dem Netzlaufwerk.
    Teilen Sie das Passwort nur mit Gruppenmitgliedern, die zwingend
    Zugriff auf diese Daten benötigen.
  \end{itemize}
\item
  \textbf{M 4.63 Sicherheitstechnische Anforderungen an den
  Telearbeitsrechner}

  \begin{itemize}
  \item
    Schützen Sie ihr Client-Benutzerkonto mit einem sicheren Passwort.
    Sperren Sie ihren Computer, wenn sie den Raum verlassen.
  \item
    Verschlüsseln Sie ihre gesamte Festplatte mittels Bitlocker oder
    Truecrypt.
  \item
    Verwenden Sie nur Passwörter, die mindestens 8 Zeichen lang sind.
  \end{itemize}
\item
  \textbf{M 4.433 Einsatz von Datenträgerverschlüsselung}

  \begin{itemize}
  \item
    Verschlüsseln Sie alle Datenträger, die Sie mobil nutzen. Etwa
    Laptop-Festplatten oder USB-Sticks (Werkzeuge: Bitlocker,
    TrustedDisk, TrueCrypt).
  \end{itemize}
\item
  \textbf{M 5.69 Schutz vor aktiven Inhalten}

  \begin{itemize}
  \item
    Konfigurieren Sie ihren Browser so, dass aktive Inhalte nur auf
    Rückfrage ausgeführt werden.
  \item
    Führen Sie aktive Inhalte, bei deren Herkunft Sie sich nicht sicher
    sind, nur in einer virtuellen Maschine oder auf einem Computer, der
    nicht an unsere IT angeschlossen ist, aus.
  \end{itemize}
\item
  \textbf{M 6.56 Datensicherung bei Einsatz kryptographischer Verfahren}

  \begin{itemize}
  \item
    Wenn Sie ein Programm zur Passwortverwaltung oder Verschlüsselung
    von Daten benutzen, denken Sie bitte daran auch dieses Programm auf
    dem Dateiserver abzulegen.
  \end{itemize}
\end{itemize}

\end{document}
